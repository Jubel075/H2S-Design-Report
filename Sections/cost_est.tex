\section{Cost Estimate}\label{sec:cost_est}

\subsection{CapEx estimate (Class 3)}

\subsubsection{Estimate classification and intent}
This estimate is prepared as an \emph{AACE Class 3} budget estimate for a process-industry project. The estimate is intended to support budget authorization and concept selection, based on a defined process configuration (BFD), preliminary design basis, and a preliminary major equipment list. Per AACE guidance, Class 3 estimates are expected to retain material uncertainty due to incomplete definition, which shall be managed through explicit contingency and a documented basis register. \cite{AACE18R97}

\subsubsection{Estimate method and cost breakdown structure}
A hybrid Class 3 method is applied:
\begin{romanlist}
\item \textbf{Equipment-based estimating (PEC):} major packages priced using budgetary quotes or allowances consistent with packaged-unit procurement.
\item \textbf{Factored bulks/indirects:} piping, E\&I, structural, installation, engineering, and commissioning estimated as factored percentages consistent with the level of definition.
\item \textbf{Contingency:} applied based on estimate maturity and key technical risks (e.g., packed-column performance uncertainty, carbonate management, and biological solids handling). \cite{AACE18R97,Flagiello2021}
\end{romanlist}

\subsubsection{Major equipment packages (budgetary basis)}
Table~\ref{tab:pec_packages} summarizes the budgetary package prices provided for this Class 3 estimate. These values are treated as \emph{budgetary allowances pending RFQ}; they must be replaced by vendor budgetary quotations during the next stage.

\begin{table}[H]
  \centering
  \caption{Purchased equipment cost (PEC) basis for major packages (budgetary allowances pending RFQ).}
  \label{tab:pec_packages}
  \renewcommand{\arraystretch}{1.3}
  \small
  \begin{tabularx}{\linewidth}{@{}X l S[table-format=6.0] c@{}}
    \toprule
    \textbf{Package} & \textbf{Scope Summary} & {\textbf{PEC (USD)}} & \textbf{Range} \\
    \midrule
    Packed Caustic Scrubber & Packed tower, internals, mist elim., skid & 75500 & $-15\% / +20\%$ \\
    Biological Oxidation & MBBR reactor, media, aeration, controls & 185000 & $-15\% / +20\%$ \\
    Solids Separation & Lamella clarifier, mix/floc, sludge pump & 89500 & $-15\% / +20\%$ \\
    \midrule
    \multicolumn{2}{@{}l}{\textbf{Subtotal (Major PEC)}} & \textbf{350000} & \\
    \bottomrule
  \end{tabularx}
\end{table}

\subsubsection{Allowances for balance-of-plant}
The BFD configuration includes additional equipment and integration scope beyond the three major packages, including inlet KO/coalescing filtration (if not existing), caustic storage and metering, neutralization tank (M-101) with acid dosing, interconnecting piping, tie-ins, and controls. These items are carried as balance-of-plant (BOP) allowances at Class 3 maturity.

\begin{table}[H]
  \centering
  \caption{Balance-of-plant (BOP) allowances (Class 3 placeholders).}
  \label{tab:bop_allowances}
  \begin{tabularx}{0.92\linewidth}{@{}lYr@{}}
    \toprule
    \textbf{Item} & \textbf{Included scope} & \textbf{Allowance (USD)} \\
    \midrule
    Inlet KO / coalescing filtration (if new) & KO drum, coalescer housing, DP indication & 60{,}000 \\
    Caustic storage and dosing & storage tank, metering pumps, pH instrumentation & 45{,}000 \\
    Neutralization tank M-101 + acid dosing & tank, mixer, dosing skid, pH loop & 55{,}000 \\
    Interconnecting piping and tie-ins & process piping, valves, supports, tie-ins & 120{,}000 \\
    Electrical and instrumentation bulks & cabling, panels, field instruments, integration & 90{,}000 \\
    \midrule
    \multicolumn{2}{@{}l}{\textbf{Subtotal (BOP allowances)}} & \textbf{370{,}000} \\
    \bottomrule
  \end{tabularx}
\end{table}

\subsubsection{Installed cost build-up (TIC)}
Total installed cost (TIC) is built up from PEC plus bulks and indirects, with contingency consistent with Class 3 maturity:
\[
  \mathrm{TIC} = \mathrm{PEC}_{\text{total}} + \mathrm{Installation/Bulks} + \mathrm{Indirects} + \mathrm{Contingency}
\]
A factored approach is used here because vendor installation manhours, detailed MTO, and IFC drawings are not available at this stage; this is aligned with the purpose of Class 3 estimating (budget authorization with incomplete detail) \cite{AACE18R97}. Installation factors are consistent with typical process industry ranges: 35\% of direct costs for field installation and bulks \cite{PetersTimmerhaus2003b,TowlerSinnott2022b}.

\begin{table}[H]
  \centering
  \caption{Class 3 total installed cost (TIC) summary (USD).}
  \label{tab:tic_summary}
  \begin{tabularx}{0.92\linewidth}{@{}lrr@{}}
    \toprule
    \textbf{Cost element} & \textbf{Base (USD)} & \textbf{Notes} \\
    \midrule
    Major PEC subtotal & 350{,}000 & Table~\ref{tab:pec_packages} \\
    BOP allowances & 370{,}000 & Table~\ref{tab:bop_allowances} \\
    \midrule
    \textbf{Total PEC (incl. BOP)} & \textbf{720{,}000} & Purchased + allowance basis \\
    Installation \& bulks factor & 650{,}000 & Factored bulks/field labor (Class 3) \\
    Indirects (EPCM, commissioning, temp.) & 280{,}000 & Factored indirects (Class 3) \\
    \midrule
    \textbf{Subtotal (before contingency)} & \textbf{1{,}650{,}000} &  \\
    Contingency (Class 3) & 330{,}000 & 20\% of subtotal \cite{AACE18R97} \\
    \midrule
    \textbf{TIC (Class 3 base)} & \textbf{1{,}980{,}000} & \textbf{USD 2.0 MM (rounded)} \\
    \bottomrule
  \end{tabularx}
\end{table}

\paragraph{Class 3 range statement.}
For reporting, the Class 3 estimate should be presented as a range around the base TIC, reflecting remaining definition uncertainty consistent with estimate classification guidance \cite{AACE18R97}. The base TIC in Table~\ref{tab:tic_summary} corresponds to an expected value; the estimate range should be refined after RFQ responses are received.

\subsection{OpEx estimate}

\subsubsection{Operating cost structure and primary drivers}
Operating cost is estimated as:
\[
  \mathrm{OpEx} = \mathrm{Chemicals} + \mathrm{Utilities} + \mathrm{Maintenance} + \mathrm{Waste} + \mathrm{Labor}
\]
For caustic scrubbing, industrial practice indicates that the dominant variable-cost uncertainty is parasitic \ce{CO2} absorption (driving caustic consumption and carbonate management) \cite{AFPM2014}. At feasibility level, chemical pricing is taken from published regional benchmarks and must be replaced with supplier quotes for the project location. \cite{IMARC_CausticSoda,IMARC_SulphuricAcid}

\begin{table}[H]
  \centering
  \caption{OpEx input assumptions (budgetary; to be replaced by supplier quotes).}
  \label{tab:opex_inputs}
  \begin{tabularx}{0.92\linewidth}{@{}lYr@{}}
    \toprule
    \textbf{Item} & \textbf{Assumption basis} & \textbf{Base value} \\
    \midrule
    \ce{NaOH} price (dry equivalent) & 2025 regional benchmark range \cite{IMARC_CausticSoda} & \SI{500}{USD/ton} \\
    Acid price (e.g., \ce{H2SO4}) & 2025 benchmark range \cite{IMARC_SulphuricAcid} & \SI{150}{USD/ton} \\
    Electricity tariff & U.S. EIA Electric Power Monthly (industrial average retail price) \cite{EIAEPM_C3} & \SI{0.10}{USD/kWh} \\
    Maintenance factor & Screening: fraction of TIC (Class 3) & 4\% of TIC / year \\
    Labor & Attended periodic operation assumption & 0.25--0.5 FTE \\
    Waste/sludge disposal & Site-specific; placeholder until vendor quote & Allowance \\
    \bottomrule
  \end{tabularx}
\end{table}

\subsection{Cost basis and assumptions}

\subsubsection{Basis register (what makes this estimate valid)}
The estimate is valid only under the following basis conditions:
\begin{romanlist}
\item The BFD configuration and battery limits remain as defined in \Cref{fig:PFD_scrub_pdf}.
\item The three major package prices in Table~\ref{tab:pec_packages} are treated as budgetary allowances pending RFQ and must be replaced by vendor budgetary quotations in the next phase.
\item Chemical price assumptions are based on published market benchmarks and must be replaced by local supplier quotes prior to any investment decision. \cite{IMARC_CausticSoda,IMARC_SulphuricAcid}
\item Contingency is applied consistent with Class 3 maturity; it does not cover scope omissions. \cite{AACE18R97}
\end{romanlist}
