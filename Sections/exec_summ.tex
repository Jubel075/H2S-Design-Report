% =========================================================
% Executive Summary
% =========================================================

\section{Executive Summary}

This report presents a comparative engineering study of caustic scrubbing technologies for sour gas sweetening at the TA-58 production facility. The objective is to enable displacement of diesel fuel consumption (estimated at \SI{500000}{\USD\per\yr}) by treating associated gas to meet the \SI{45}{ppmv} \ce{H2S} specification for the installed Caterpillar G3408 gas engines.

\subsection{Design Basis}

The study evaluates treatment of \textbf{660,000~SCFD} associated gas containing \SI{1800}{ppmv} \ce{H2S} and \SI{2.86}{\percent} \ce{CO2}. The required removal efficiency is 97.5\% to achieve the engine fuel specification. At this scale:
\begin{itemize}
\item \ce{H2S} load: 0.056~kmol/h (1.9~kg/h)
\item \ce{CO2} load: 0.90~kmol/h (39~kg/h)
\item Molar ratio \ce{CO2}/\ce{H2S}: approximately 16:1
\end{itemize}

The 16-fold excess of \ce{CO2} creates the central design challenge: non-selective absorption would consume caustic at rates that render the project economically non-viable.

\subsection{Technology Comparison}

Two caustic scrubbing configurations are evaluated:

\begin{table}[H]
\centering
\caption{Technology comparison summary at 660,000 SCFD}
\label{tab:exec_options}
\small
\begin{tabularx}{\textwidth}{@{}lYY@{}}
\toprule
\textbf{Parameter} & \textbf{Packed Column} & \textbf{SCT Static Mixer} \\
\midrule
Equipment size & 0.75~m dia. $\times$ 5.5~m height & 0.3~m $\times$ 1.5~m inline \\
\ce{H2S} removal & 97.5\% & $>$98\% \\
\ce{CO2} co-absorption & 50--80\% (equilibrium) & 2--10\% (kinetic control) \\
\ce{NaOH} consumption & 300--500~ton/yr & 46--90~ton/yr \\
\ce{NaOH} cost & \$150,000--250,000/yr & \$23,000--45,000/yr \\
Technology maturity & Well-established & Emerging (growing field data) \\
\bottomrule
\end{tabularx}
\end{table}

The SCT technology exploits the $10^6$ difference in reaction rate constants between \ce{H2S} and \ce{CO2}, achieving selective removal through controlled short contact time (50--200~ms) rather than relying on pH control.

\subsection{Economic Assessment}

The economic analysis reveals a critical finding: \textbf{at the 660,000~SCFD scale, selectivity determines project viability.}

\begin{table}[H]
\centering
\caption{Economic comparison (base case)}
\label{tab:exec_economics}
\begin{tabular}{@{}lccc@{}}
\toprule
\textbf{Indicator} & \textbf{Packed Column} & \textbf{SCT System} & \textbf{Assessment} \\
\midrule
Capital cost (TIC) & \$2.53M & \$1.78M & SCT 30\% lower \\
Net annual benefit & \$211,000/yr & \$417,000/yr & SCT 2$\times$ higher \\
Simple payback & 12.0 years & 4.3 years & SCT 2.8$\times$ faster \\
IRR (15-year) & 5.3\% & 22.8\% & SCT exceeds hurdle \\
NPV at 12\% & --\$87,000 & +\$1,060,000 & SCT creates value \\
\bottomrule
\end{tabular}
\end{table}

\textbf{Key conclusions:}
\begin{itemize}
\item The \textbf{packed column fails to meet the 15\% IRR hurdle rate} at base case assumptions due to high caustic consumption from \ce{CO2} co-absorption.
\item The \textbf{SCT system exceeds the hurdle rate} (22.8\% IRR) and delivers positive NPV of \$1.06M over the 15-year project life.
\item The economic cost of non-selective \ce{CO2} absorption is approximately \$1.15M in lifetime value destruction.
\end{itemize}

\subsection{Technology Recommendation}

Based on the quantitative analysis, \textbf{SCT static mixer technology is recommended} for the TA-58 application. The recommendation is supported by:
\begin{romanlist}
\item Superior economics: 5--10$\times$ lower caustic consumption provides \$200,000/yr operating cost advantage
\item Inherent selectivity: Design-based selectivity reduces operational risk versus pH-controlled selectivity
\item Compact footprint: Inline installation minimises plot space requirements
\item Growing field experience: SCT caustic scrubbers have demonstrated performance in refinery applications
\end{romanlist}

The packed column remains a viable fallback if vendor engagement identifies concerns with SCT technology availability, but the economic penalty should be explicitly considered.

\subsection{Critical Uncertainties}

Resolution of the following is required before project sanction:
\begin{enumerate}
\item \textbf{Vendor quotations:} Budgetary quotes from 2--3 SCT suppliers to validate capital estimate
\item \textbf{Diesel displacement:} Historical fuel records to confirm \$500,000/yr benefit
\item \textbf{Selectivity guarantee:} Vendor performance guarantees for \ce{CO2} co-absorption limits
\item \textbf{Spent caustic treatment:} MBBR vendor confirmation for 43~kg~S/day capacity
\end{enumerate}

\subsection{Recommended Next Steps}

\begin{enumerate}
\item Issue RFQ to SCT caustic scrubber vendors (e.g., Merichem, Trimeric, Koch-Glitsch)
\item Validate diesel displacement through operational data analysis
\item Conduct HAZID workshop for hazard identification
\item If required, plan pilot testing to validate selectivity under field conditions
\item Following vendor responses, proceed to detailed feasibility and FEED
\end{enumerate}

This study demonstrates that selective \ce{H2S} removal using SCT technology provides a technically feasible and economically attractive solution for the TA-58 sour gas treatment. The project creates an estimated \$1.06M NPV over 15 years while enabling diesel displacement and emissions reduction.

