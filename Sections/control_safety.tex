% =========================================================
% Control Philosophy and Safety
% =========================================================

\section{Control Philosophy and Safety}
\label{sec:control_safety}

\subsection{Control Philosophy Overview}

The control system for the caustic scrubber must achieve two objectives: (i) maintain treated gas quality at $<$\SI{45}{ppmv} \ce{H2S}, and (ii) manage caustic consumption by controlling \ce{CO2} co-absorption. For the SCT technology, the second objective is achieved through inherent design (short contact time), while pH control ensures the first objective is reliably met.

Figure~\ref{fig:control_loops} illustrates the control architecture, comprising two primary control loops and two critical safety interlocks.

\begin{figure}[H]
\centering
\begin{tikzpicture}[
    scale=0.85,
    font=\small,
    >=Stealth,
    line width=0.7pt,
    block/.style={rectangle, draw, thick, minimum width=2cm, minimum height=0.8cm, align=center, font=\scriptsize},
    smallblock/.style={rectangle, draw, thick, minimum width=1.5cm, minimum height=0.6cm, align=center, font=\scriptsize},
    sensor/.style={circle, draw, thick, minimum size=0.6cm, font=\scriptsize},
]

% ===== CONTROL LOOP 1: pH Control =====
\node[font=\footnotesize\bfseries] at (0,6.5) {Loop 1: pH Control};

\node[block, fill=MainBlue!10] (sp1) at (-3,5.5) {pH Setpoint\\10--11 (SCT)};
\node[block, fill=white] (pic) at (0,5.5) {pH Controller\\(PI)};
\node[block, fill=AccentA!20] (valve1) at (3,5.5) {NaOH\\Valve};
\node[block, fill=gray!10] (proc1) at (6,5.5) {Scrubber\\Process};
\node[sensor, fill=AccentB!20] (ph) at (6,4.2) {pH};

\draw[->, thick] (sp1) -- (pic);
\draw[->, thick] (pic) -- (valve1);
\draw[->, thick] (valve1) -- (proc1);
\draw[->, thick] (proc1) -- (ph);
\draw[->, thick] (ph) -| (0,4.2) -- (pic);

% ===== CONTROL LOOP 2: Level Control =====
\node[font=\footnotesize\bfseries] at (0,3.2) {Loop 2: Level Control};

\node[block, fill=MainBlue!10] (sp2) at (-3,2.2) {Level Setpoint\\50\%};
\node[block, fill=white] (lic) at (0,2.2) {Level Controller\\(PI)};
\node[block, fill=AccentA!20] (valve2) at (3,2.2) {Blowdown\\Valve};
\node[block, fill=gray!10] (proc2) at (6,2.2) {Caustic\\Tank};
\node[sensor, fill=AccentB!20] (lt) at (6,0.9) {LT};

\draw[->, thick] (sp2) -- (lic);
\draw[->, thick] (lic) -- (valve2);
\draw[->, thick] (valve2) -- (proc2);
\draw[->, thick] (proc2) -- (lt);
\draw[->, thick] (lt) -| (0,0.9) -- (lic);

% ===== INTERLOCK 1: H2S Breakthrough =====
\node[font=\footnotesize\bfseries] at (10.5,6.5) {Interlock 1: H2S Breakthrough};

\node[sensor, fill=AccentC!30] (at) at (8.5,5.5) {AT};
\node[smallblock, fill=AccentC!10] (logic1) at (11,5.5) {>45 ppmv?};
\node[smallblock, fill=AccentC!30] (action1) at (14,5.5) {Divert to\\flare};

\draw[->, thick] (at) -- (logic1);
\draw[->, thick, AccentC] (logic1) -- node[above, font=\tiny] {YES} (action1);
\draw[->, dashed, gray] (logic1) -- ++(0,-0.8) node[below, font=\tiny] {NO};

% ===== INTERLOCK 2: Low Caustic =====
\node[font=\footnotesize\bfseries] at (10.5,3.2) {Interlock 2: Low Caustic Inventory};

\node[sensor, fill=AccentC!30] (lt2) at (8.5,2.2) {LT};
\node[smallblock, fill=AccentC!10] (logic2) at (11,2.2) {<20\%?};
\node[smallblock, fill=AccentC!30] (action2) at (14,2.2) {Alarm +\\stop pump};

\draw[->, thick] (lt2) -- (logic2);
\draw[->, thick, AccentC] (logic2) -- node[above, font=\tiny] {YES} (action2);
\draw[->, dashed, gray] (logic2) -- ++(0,-0.8) node[below, font=\tiny] {NO};

% Legend
\node[font=\scriptsize, anchor=west] at (-3.5,-0.2) {\textbf{Legend:}};
\fill[MainBlue!10] (-3.5,-0.6) rectangle (-3.0,-0.3);
\draw (-3.5,-0.6) rectangle (-3.0,-0.3);
\node[font=\tiny, anchor=west] at (-2.9,-0.45) {Setpoint};
\fill[AccentB!20] (-1.5,-0.6) rectangle (-1.0,-0.3);
\draw (-1.5,-0.6) rectangle (-1.0,-0.3);
\node[font=\tiny, anchor=west] at (-0.9,-0.45) {Sensor};
\fill[AccentC!30] (0.5,-0.6) rectangle (1.0,-0.3);
\draw (0.5,-0.6) rectangle (1.0,-0.3);
\node[font=\tiny, anchor=west] at (1.1,-0.45) {Safety interlock};

\end{tikzpicture}
\caption{Control philosophy block diagram showing primary control loops (pH and level) and safety interlocks (\ce{H2S} breakthrough and low caustic inventory). The pH control loop is critical for SCT operation, requiring $\pm$0.3~pH unit tolerance.}
\label{fig:control_loops}
\end{figure}


\subsection{Primary Control Loops}

Table~\ref{tab:control_loops} summarises the essential control loops for stable scrubber operation.

\begin{table}[H]
\centering
\caption{Primary control loop specifications}
\label{tab:control_loops}
\small
\begin{tabularx}{\textwidth}{@{}llYc@{}}
\toprule
\textbf{Loop} & \textbf{Controller} & \textbf{Purpose} & \textbf{Setpoint} \\
\midrule
pH control & PI (tuned for fast response) & Maintain solvent alkalinity for complete \ce{H2S} removal; for SCT, tighter control prevents over-absorption of \ce{CO2} & 10--11 (SCT) \\
Level control & PI (slow integral) & Manage sulfide/carbonate inventory in recirculation tank; prevent accumulation & 50\% \\
Recirculation flow & Manual or ratio & Maintain design L/G ratio and ensure adequate wetting of contactor & 2--5$\times$ stoichiometric \\
\bottomrule
\end{tabularx}
\end{table}

\subsubsection{pH Control for SCT Operation}

For the SCT option, pH control is the critical variable for maintaining selectivity. The operating philosophy differs from conventional packed column operation:

\begin{itemize}
\item \textbf{Setpoint:} pH 10--11 (lower than packed column's pH 12--13)
\item \textbf{Tolerance:} $\pm$0.3 pH units (tighter than packed column's $\pm$1.0)
\item \textbf{Response time:} Fast PI tuning required; pH must respond within 30~s to feed changes
\item \textbf{Measurement:} Industrial pH electrode with automatic temperature compensation; redundant sensors recommended
\end{itemize}

The lower pH setpoint reduces hydroxide availability, which further limits \ce{CO2} reaction kinetics while maintaining sufficient driving force for the instantaneous \ce{H2S} reaction.

\subsubsection{Level Control and Blowdown Strategy}

The blowdown rate determines the steady-state sulfide and carbonate concentrations in the recirculating caustic:

\begin{equation}
  C_{\text{sulfide,ss}} = \frac{\dot{n}_{\ce{H2S}}}{F_{\text{blowdown}}}
\end{equation}

For the SCT system at 5\% \ce{CO2} absorption:
\begin{itemize}
\item Target blowdown rate: 50--100~L/h
\item Sulfide concentration: 2--4~wt\% (as \ce{NaHS})
\item Carbonate concentration: $<$1~wt\%
\end{itemize}

The low carbonate content simplifies downstream biological treatment compared to the packed column option.

\subsection{Critical Safety Interlocks}

Two safety interlocks are required for engine fuel protection and personnel safety:

\subsubsection{Interlock 1: \ce{H2S} Breakthrough}

\begin{table}[H]
\centering
\caption{H2S breakthrough interlock specification}
\label{tab:interlock_h2s}
\begin{tabular}{@{}ll@{}}
\toprule
\textbf{Parameter} & \textbf{Specification} \\
\midrule
Measurement & Continuous \ce{H2S} analyser (electrochemical or IR) \\
Location & Treated gas outlet, downstream of demister \\
High alarm & \SI{30}{ppmv} (operator warning) \\
High-high alarm \& action & \SI{45}{ppmv} (automatic diversion) \\
Action & Close fuel gas valve; open flare diversion valve \\
Reset & Manual (requires operator confirmation after investigation) \\
\bottomrule
\end{tabular}
\end{table}

The interlock ensures that off-spec gas cannot reach the engine fuel header. Upon \ce{H2S} breakthrough:
\begin{enumerate}
\item Treated gas is diverted to the flare header
\item Alarm annunciates in control room
\item Operator investigates root cause (low caustic, pH deviation, flow upset)
\item System remains diverted until manual reset after investigation
\end{enumerate}

\subsubsection{Interlock 2: Low Caustic Inventory}

\begin{table}[H]
\centering
\caption{Low caustic inventory interlock specification}
\label{tab:interlock_caustic}
\begin{tabular}{@{}ll@{}}
\toprule
\textbf{Parameter} & \textbf{Specification} \\
\midrule
Measurement & Level transmitter on caustic storage tank \\
Low alarm & 30\% (operator warning) \\
Low-low alarm \& action & 20\% (inhibit dosing pump) \\
Action & Inhibit \ce{NaOH} dosing pump start; alarm \\
Purpose & Prevent dry running; ensure caustic availability \\
\bottomrule
\end{tabular}
\end{table}

\subsection{Instrumentation Requirements}

Table~\ref{tab:instrumentation} lists the minimum instrumentation for safe operation.

\begin{table}[H]
\centering
\caption{Minimum instrumentation list}
\label{tab:instrumentation}
\small
\begin{tabular}{@{}llll@{}}
\toprule
\textbf{Tag} & \textbf{Description} & \textbf{Type} & \textbf{Purpose} \\
\midrule
AIT-001 & Outlet \ce{H2S} analyser & Electrochemical/IR & Product quality; interlock \\
pHIT-001 & Recirculation pH & Industrial electrode & Control; selectivity \\
LIT-001 & Recirculation tank level & Differential pressure & Level control \\
LIT-002 & Caustic storage level & Radar/ultrasonic & Inventory; interlock \\
FIT-001 & Sour gas inlet flow & Orifice plate & Mass balance \\
FIT-002 & \ce{NaOH} dosing flow & Magnetic flowmeter & Consumption tracking \\
PdIT-001 & Contactor differential pressure & DP cell & Fouling/flooding detection \\
\bottomrule
\end{tabular}
\end{table}

\subsection{Safety Considerations}

\ce{H2S} is immediately dangerous to life at concentrations above \SI{100}{ppmv} \cite{OSHAH2S,NIOSHH2S}. The scrubber system handles gas streams at \SI{1800}{ppmv} inlet concentration. Minimum safety provisions:

\begin{itemize}
\item Fixed \ce{H2S} detectors at scrubber inlet/outlet and pump areas (alarm at \SI{10}{ppmv}, evacuate at \SI{20}{ppmv})
\item Integration with site emergency shutdown system
\item Respiratory protection (SCBA) available within egress distance
\item Wind sock/indicator at scrubber location
\item Caustic handling per site chemical management procedures (PPE: face shield, chemical-resistant gloves, apron)
\end{itemize}

\subsection{HAZOP Scope}

Formal HAZOP study during detailed engineering will address:
\begin{romanlist}
\item Deviation analysis for all process nodes (feed, scrubber, caustic system, treated gas)
\item Safeguard verification for identified hazards
\item Safety integrity level (SIL) determination for \ce{H2S} breakthrough interlock
\item Operability issues including startup, shutdown, and upset conditions
\item Human factors for operator interface design
\end{romanlist}

The conceptual design provides sufficient definition for meaningful HAZOP review. The control philosophy outlined in this section establishes the framework for safe operation pending detailed engineering validation.

