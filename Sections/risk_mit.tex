% =========================================================
% Risk Assessment
% =========================================================

\section{Risk Assessment}
\label{sec:risk_mit}

\subsection{Key Technical Risks}

Three primary risks require management during project advancement:

\begin{table}[H]
\centering
\caption{Primary risk register}
\label{tab:key_risks}
\small
\begin{tabularx}{\textwidth}{@{}lccY@{}}
\toprule
\textbf{Risk} & \textbf{L} & \textbf{C} & \textbf{Mitigation} \\
\midrule
\ce{H2S} breakthrough & 2 & 3 & Conservative design margin; continuous outlet \ce{H2S} monitoring with fuel-divert interlock \\
\addlinespace
\ce{CO2} co-absorption variability & 3 & 2 & Evaluate selective operation via simulation; SCT option inherently provides kinetic selectivity \\
\addlinespace
Biological treatment scale & 2 & 2 & Vendor feasibility confirmation for 43~kg~S/day capacity; fallback to chemical oxidation if MBBR not viable \\
\bottomrule
\end{tabularx}
\end{table}

Likelihood (L): 1=Rare, 2=Possible, 3=Likely. Consequence (C): 1=Minor, 2=Moderate, 3=Major.

\subsection{HSE Considerations}

\ce{H2S} toxicity and \ce{NaOH} chemical burns represent the primary HSE hazards. Minimum safeguards include:
\begin{itemize}
\item Fixed \ce{H2S} gas detection integrated with site DCS
\item Secondary containment for caustic storage areas
\item Safety showers within \SI{10}{m} of process areas
\item Respiratory protection program and PPE training
\end{itemize}

Formal HAZID/HAZOP studies are required in the detailed engineering phase to identify additional site-specific hazards.

\subsection{Risk Closure Requirements}

Prior to detailed engineering, the following activities are required:
\begin{romanlist}
\item Process simulation with electrolyte thermodynamics to refine \ce{CO2} absorption kinetics
\item Vendor engagement to validate mass transfer performance data
\item Confirmation of biological treatment feasibility at design sulfide loading
\item Diesel benefit validation through operational data analysis
\end{romanlist}

Residual risks (packing performance uncertainty $\pm$20\%, \ce{CO2} co-absorption variability $\pm$30\%) are acceptable provided project IRR remains above 15\% hurdle under sensitivity analysis.
