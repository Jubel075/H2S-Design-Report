% =========================================================
% Section 2: Introduction and Design Basis
% =========================================================

\section{Introduction and Design Basis}
\label{sec:introduction}

% Essential nomenclature entries
\nomenclature[A]{SCFD}{Standard Cubic Feet per Day}
\nomenclature[A]{VRU}{Vapor Recovery Unit}
\nomenclature[A]{MBBR}{Moving Bed Biofilm Reactor}
\nomenclature[A]{SCT}{Short Contact Time}
\nomenclature[A]{IRR}{Internal Rate of Return}
\nomenclature[A]{NPV}{Net Present Value}
\nomenclature[A]{TIC}{Total Installed Cost}
\nomenclature[A]{HETP}{Height Equivalent to a Theoretical Plate}
\nomenclature[P]{$N_{TU}$}{Number of Transfer Units (dimensionless)}
\nomenclature[P]{$HTU$}{Height of a Transfer Unit (\si{m})}
\nomenclature[P]{Ha}{Hatta number (dimensionless)}
\nomenclature[P]{Da}{Damk\"ohler number (dimensionless)}
\nomenclature[P]{$S$}{Selectivity (dimensionless)}
\nomenclature[P]{$E$}{Enhancement factor (dimensionless)}

\subsection{Problem Definition}

TA-58 produces a methane-rich associated gas stream containing \SI{1800}{ppmv} \ce{H2S}, exceeding the \SI{45}{ppmv} fuel specification for the installed Caterpillar G3408 (HPG) dual-gas engines \cite{TA58Dataset}. Under current practice, most associated gas is routed to flare while the pump station relies on diesel at an estimated \SI{500000}{\USD\per\yr}. Implementation of an \ce{H2S} removal system enables: (i) on-spec fuel gas for engine operation; (ii) reduced routine flaring and \ce{SO2} emissions; and (iii) diesel displacement with corresponding \ce{CO2} and cost reductions.

The gas stream also contains \SI{2.86}{\percent} \ce{CO2} (\SI{28550}{ppmv}), which creates a selectivity challenge: non-selective absorption would consume caustic at rates that render the project economically unviable. As Pudi et al.\ \cite{Pudi2022} note in their comprehensive review of \ce{H2S} capture technologies, the presence of \ce{CO2} fundamentally influences technology selection and operating strategy for sour gas treatment.

\subsection{Study Scope}

This study evaluates the technical and economic feasibility of treating the TA-58 sour associated gas to engine fuel-gas specification. The scope comprises:
\begin{romanlist}
  \item feed characterisation and design basis establishment;
  \item comparative evaluation of applicable \ce{H2S} removal technologies;
  \item conceptual equipment sizing with material balances;
  \item techno-economic assessment against the diesel displacement baseline;
  \item integration requirements with existing systems.
\end{romanlist}

Exclusions: detailed engineering, vendor guarantees, formal process safety studies (HAZID/HAZOP/LOPA), procurement, permitting, and commissioning support.

\subsection{Design Basis}

The key design inputs are summarised in Table~\ref{tab:design_basis_intro}, based on operational data from the TA-58 facility \cite{TA58Dataset}.

\begin{table}[H]
\centering
\caption{Design basis parameters}
\label{tab:design_basis_intro}
\begin{tabular}{@{}lcc@{}}
\toprule
\textbf{Parameter} & \textbf{Value} & \textbf{Units} \\
\midrule
Gas source & VRU discharge & --- \\
Flow rate (standard) & 660,000 & SCFD \\
Flow rate (SI equivalent) & 779 & Nm$^3$/h \\
Total molar flow & 31.3 & kmol/h \\
Inlet \ce{H2S} & 1,800 & ppmv \\
Inlet \ce{CO2} & 28,550 & ppmv (2.86\%) \\
Required outlet \ce{H2S} & $<45$ & ppmv \\
Removal efficiency & $>97.5$ & \% \\
Operating temperature & 25--40 & \si{\celsius} \\
Operating pressure & Atmospheric & --- \\
\bottomrule
\end{tabular}
\end{table}

At the design flow rate, the component molar flows are:
\begin{itemize}
\item \textbf{\ce{H2S} load:} $31.3 \times 0.0018 = \SI{0.056}{kmol/h}$ (\SI{1.9}{kg/h} or \SI{46}{kg/day})
\item \textbf{\ce{CO2} load:} $31.3 \times 0.0286 = \SI{0.90}{kmol/h}$ (\SI{39}{kg/h})
\item \textbf{Molar ratio \ce{CO2}/\ce{H2S}:} $\approx 16:1$
\end{itemize}

The 16-fold excess of \ce{CO2} relative to \ce{H2S} creates the central design challenge: if \ce{CO2} is absorbed non-selectively alongside \ce{H2S}, caustic consumption becomes economically prohibitive (see Section~\ref{sec:technology_options}).

\subsection{Technology Screening Summary}

The screening evaluation considered technologies capable of reducing \ce{H2S} from \SI{1800}{ppmv} to $<\SI{45}{ppmv}$ (approximately 97.5\% removal). Caustic scrubbing was selected for detailed evaluation based on operability at low pressure, tolerance to liquid carryover, and potential for sulfide stabilisation via biological treatment \cite{Pudi2022}. Alternative technologies (membrane + ZnO, liquid redox, chemical scavengers) were not advanced due to technical or economic limitations at the site scale.

A key design consideration is the presence of \ce{CO2}: the rate constant for \ce{CO2} absorption in caustic is approximately $10^6$ times slower than for \ce{H2S}, which can be exploited through contactor design to achieve selective removal and limit caustic consumption \cite{AFPM2014,Danckwerts1965,astarita1983}.

Figure~\ref{fig:h2s-logic-clean} illustrates the design logic flow from feed characterisation through to operability limits, highlighting the critical linkages between feed conditions, solvent selection, and equipment sizing.

\begin{figure}[H]
  \centering
  \fbox{\resizebox{0.98\linewidth}{!}{%
      \begin{tikzpicture}[node distance=11mm and 36mm]

        %------------------------------------------------
        % Column anchors
        %------------------------------------------------
        \coordinate (ColL) at (0,0);
        \coordinate (ColR) at (7.0,0);
        \coordinate (LaneX) at (10.0,0);

        \def\RowGap{1.65}

        %------------------------------------------------
        % Vertical transition spine
        %------------------------------------------------
        \draw[sidearrow]
        ($(LaneX)+(0,0.3)$) -- node[stepside,pos=0.5]{$5$}($(LaneX)+(0,-3.25*\RowGap)$);

        %------------------------------------------------
        % ROW 1
        %------------------------------------------------
        \node[stage] (feed) at (ColL) {Feed Gas};
        \node[stage] (perf) at (ColR) {Performance};

        \node[hint] [below=1.8mm of feed] {Flow, \ce{H2S}/\ce{CO2}, $T$, $P$ define duty};
        \node[hint] [below=1.8mm of perf] {Target outlet \ce{H2S} and removal};

        \draw[mainarrow]
        ([xshift=1mm]feed.east) --
        node[stepmain, pos=0.5]{1}
        ([xshift=-1mm]perf.west);

        %------------------------------------------------
        % ROW 2
        %------------------------------------------------
        \node[stage] (chem) at ($(ColL)+(0,-\RowGap)$) {Reaction\\Chemistry};
        \node[stage] (solv) at ($(ColR)+(0,-\RowGap)$) {Solvent\\Design};

        \node[hint] [below=1.8mm of chem]
        {\ce{H2S}+\ce{NaOH} $\rightarrow$ \ce{NaHS}/\ce{Na2S}; \ce{CO2} side reactions};
        \node[hint] [below=1.8mm of solv]
        {Select wt\% \ce{NaOH}, pH, circulation/makeup};

        \draw[mainarrow]
        ([xshift=1mm]chem.east) --
        node[stepmain, pos=0.5]{2}
        ([xshift=-1mm]solv.west);

        %------------------------------------------------
        % ROW 3
        %------------------------------------------------
        \node[stage] (hyd) at ($(ColL)+(0,-2*\RowGap)$) {Hydraulics};
        \node[stage] (mt)  at ($(ColR)+(0,-2*\RowGap)$) {Mass\\Transfer};

        \node[hint] [below=1.8mm of hyd]
        {Size $D_T$, flooding margin, and control $\Delta P$};
        \node[hint] [below=1.8mm of mt]
        {Packing; $a$, $k_La/k_Ga$; HTU/NTU and height};

        \draw[mainarrow]
        ([xshift=1mm]hyd.east) --
        node[stepmain, pos=0.5]{3}
        ([xshift=-1mm]mt.west);

        %------------------------------------------------
        % ROW 4
        %------------------------------------------------
        \node[stage] (ops) at ($(ColL)+(0,-3*\RowGap)$) {Operations};
        \node[stage] (lim) at ($(ColR)+(0,-3*\RowGap)$) {Technology\\Limits};

        \node[hint] [below=1.8mm of ops]
        {$L/V$, distribution, \ce{NaOH} use, \ce{CO2} co-absorption};
        \node[hint] [below=1.8mm of lim]
        {Non-regenerative caustic: economics, blowdown, effluent handling};

        \draw[mainarrow]
        ([xshift=1mm]ops.east) --
        node[stepmain, pos=0.5]{4}
        ([xshift=-1mm]lim.west);

      \end{tikzpicture}
  }} % end resizebox
  \caption{Design logic map for caustic \ce{H2S} scrubbing. The diagram links feed conditions and performance targets to solvent selection, hydraulic sizing, mass-transfer design, and operability limits.}
  \label{fig:h2s-logic-clean}
\end{figure}


\subsection{Decision Criteria}

Project progression is recommended when:
\begin{romanlist}
  \item treated gas meets \SI{45}{ppmv} \ce{H2S} specification;
  \item a practical sulfur-handling route is defined;
  \item screening-level risk assessment indicates manageable hazards;
  \item positive economics relative to diesel displacement at AACE Class~3 maturity;
  \item tie-ins are feasible within site constraints.
\end{romanlist}
