% =========================================================
% Economic Evaluation
% =========================================================

\section{Economic Evaluation}
\label{sec:eco_eval}

The economic viability of the TA-58 sour gas project depends critically on the balance between capital investment, operating costs (dominated by caustic consumption), and the diesel displacement benefit. At the 660,000~SCFD scale, selectivity---the ability to remove \ce{H2S} while minimising \ce{CO2} co-absorption---determines whether the project creates or destroys value.

\subsection{Economic Framework}

\subsubsection{Fixed Benefit}

The diesel displacement benefit is approximately \textbf{\$500,000/yr}, based on the fuel consumption of two Caterpillar G3408 dual-fuel engines \cite{CaterpillarG3408}. This benefit is independent of gas flow rate (the engines consume a fixed amount of fuel) and represents the maximum possible annual saving.

\subsubsection{Variable Costs}

Operating costs are dominated by:
\begin{itemize}
\item \textbf{Caustic consumption:} \$500/ton delivered (50\% \ce{NaOH} solution) \cite{IMARC_CausticSoda}
\item \textbf{Biological treatment:} \$10--20/kg~S for MBBR operation
\item \textbf{Utilities:} Electricity for pumps, blowers (estimated \$15,000--25,000/yr)
\item \textbf{Maintenance:} 3\% of TIC annually
\end{itemize}

\subsection{Operating Cost Comparison}

Table~\ref{tab:opex_full} presents the complete operating cost comparison for both technologies across the range of plausible \ce{CO2} absorption scenarios.

\begin{table}[H]
\centering
\caption{Annual operating cost comparison at 660,000 SCFD}
\label{tab:opex_full}
\small
\begin{tabular}{@{}lcccc@{}}
\toprule
\textbf{Technology / Scenario} & \textbf{NaOH} & \textbf{NaOH Cost} & \textbf{Other OpEx} & \textbf{Total OpEx} \\
 & \textbf{(ton/yr)} & \textbf{(\$/yr)} & \textbf{(\$/yr)} & \textbf{(\$/yr)} \\
\midrule
\multicolumn{5}{l}{\textit{Packed Column}} \\
\quad Conservative (50\% \ce{CO2}) & 298 & 149,000 & 75,000 & 224,000 \\
\quad Base case (70\% \ce{CO2}) & 418 & 209,000 & 80,000 & 289,000 \\
\quad High \ce{CO2} (90\%) & 538 & 269,000 & 85,000 & 354,000 \\
\midrule
\multicolumn{5}{l}{\textit{SCT Static Mixer}} \\
\quad Optimistic (2\% \ce{CO2}) & 32 & 16,000 & 55,000 & 71,000 \\
\quad Base case (5\% \ce{CO2}) & 46 & 23,000 & 60,000 & 83,000 \\
\quad Conservative (10\% \ce{CO2}) & 75 & 38,000 & 65,000 & 103,000 \\
\bottomrule
\end{tabular}
\end{table}

\subsubsection{Net Annual Benefit}

The net annual benefit is calculated as diesel displacement minus total operating cost:

\begin{table}[H]
\centering
\caption{Net annual benefit comparison}
\label{tab:net_benefit}
\begin{tabular}{@{}lcc@{}}
\toprule
\textbf{Scenario} & \textbf{Total OpEx (\$/yr)} & \textbf{Net Benefit (\$/yr)} \\
\midrule
\multicolumn{3}{l}{\textit{Packed Column}} \\
\quad Best case (50\% \ce{CO2}) & 224,000 & \textbf{276,000} \\
\quad Base case (70\% \ce{CO2}) & 289,000 & \textbf{211,000} \\
\quad High \ce{CO2} (90\%) & 354,000 & \textbf{146,000} \\
\midrule
\multicolumn{3}{l}{\textit{SCT Static Mixer}} \\
\quad Optimistic (2\% \ce{CO2}) & 71,000 & \textbf{429,000} \\
\quad Base case (5\% \ce{CO2}) & 83,000 & \textbf{417,000} \\
\quad Conservative (10\% \ce{CO2}) & 103,000 & \textbf{397,000} \\
\bottomrule
\end{tabular}
\end{table}

\textbf{Key finding:} The SCT option delivers \textbf{1.5--3$\times$ higher net annual benefit} than the packed column across all scenarios. This \$140,000--280,000/yr difference compounds over the project life.

\subsection{Capital Cost Estimates}

Capital costs are estimated using the factored approach per AACE Class~4/5 methodology \cite{AACE18R97}, scaled from industry benchmarks \cite{TowlerSinnott2022b,PetersTimmerhaus2003b}. Installation factors (35\% of direct costs) are consistent with typical process industry ranges \cite{PetersTimmerhaus2003b}.

\begin{table}[H]
\centering
\caption{Capital cost estimates (AACE Class~4)}
\label{tab:capex}
\begin{tabular}{@{}lcc@{}}
\toprule
\textbf{Cost Element} & \textbf{Packed Column} & \textbf{SCT System} \\
\midrule
Scrubber/contactor & \$450,000 & \$180,000 \\
Caustic storage \& dosing & \$150,000 & \$120,000 \\
Spent caustic treatment (MBBR) & \$350,000 & \$300,000 \\
Instrumentation \& controls & \$180,000 & \$200,000 \\
Piping \& bulks & \$220,000 & \$150,000 \\
\midrule
\textbf{Direct costs} & \textbf{1,350,000} & \textbf{950,000} \\
\midrule
Engineering \& procurement (15\%) & 203,000 & 143,000 \\
Installation (35\%) & 473,000 & 333,000 \\
Contingency (25\%) & 507,000 & 357,000 \\
\midrule
\textbf{Total Installed Cost (TIC)} & \textbf{\$2,530,000} & \textbf{\$1,780,000} \\
\bottomrule
\end{tabular}
\end{table}

The SCT system has \textbf{30\% lower capital cost} due to the compact contactor and simpler auxiliary systems.

\subsection{Economic Performance Indicators}

Using the base case assumptions (packed column at 70\% \ce{CO2}, SCT at 5\% \ce{CO2}), the economic indicators are:

\begin{table}[H]
\centering
\caption{Economic performance summary (15-year life, 12\% discount rate). See \Cref{app:economic_model} for NPV/IRR methodology.}
\label{tab:eco_indicators}
\begin{tabular}{@{}lccc@{}}
\toprule
\textbf{Indicator} & \textbf{Packed Column} & \textbf{SCT System} & \textbf{Advantage} \\
\midrule
Net annual benefit & \$211,000/yr & \$417,000/yr & SCT +\$206k/yr \\
Simple payback & 12.0 years & 4.3 years & SCT 2.8$\times$ faster \\
IRR (15-year) & 5.3\% & 22.8\% & SCT +17.5 pp \\
NPV at 12\% & --\$87,000 & +\$1,060,000 & SCT +\$1.15M \\
\bottomrule
\end{tabular}
\end{table}

\textbf{Critical observations:}
\begin{romanlist}
\item The \textbf{packed column fails to meet the 15\% IRR hurdle rate} at base case assumptions (70\% \ce{CO2} absorption). Only at the optimistic 50\% \ce{CO2} scenario does IRR approach 10\%.

\item The \textbf{SCT system exceeds the hurdle rate by a comfortable margin} (22.8\% IRR vs 15\% hurdle) and delivers positive NPV of \$1.06M.

\item The difference in lifetime value creation is approximately \textbf{\$1.15M}---the economic cost of non-selective \ce{CO2} absorption.
\end{romanlist}

\subsection{Sensitivity Analysis}

\subsubsection{Parametric Sensitivity}

Table~\ref{tab:sensitivity} shows IRR sensitivity to key parameters for both technologies.

\begin{table}[H]
\centering
\caption{IRR sensitivity analysis (15-year life)}
\label{tab:sensitivity}
\small
\begin{tabular}{@{}lccccc@{}}
\toprule
\textbf{Parameter} & \textbf{Range} & \multicolumn{2}{c}{\textbf{Packed Column}} & \multicolumn{2}{c}{\textbf{SCT System}} \\
 & & Low & High & Low & High \\
\midrule
\ce{CO2} absorption & 50--90\% / 2--10\% & 10.2\% & 1.8\% & 24.3\% & 21.0\% \\
Capital cost & $\pm$30\% & 9.5\% & 2.3\% & 30.1\% & 17.4\% \\
Diesel benefit & \$400k--\$600k/yr & 1.2\% & 9.3\% & 17.6\% & 27.7\% \\
NaOH price & \$400--\$600/ton & 6.8\% & 3.7\% & 23.5\% & 22.0\% \\
\bottomrule
\end{tabular}
\end{table}

\subsubsection{Selectivity Threshold Analysis}

A critical question is: \textit{What level of \ce{CO2} selectivity is required for the packed column to be viable?}

Setting the NPV = 0 threshold (breakeven) and solving for the required \ce{CO2} absorption percentage:
\begin{equation}
  \text{Maximum \ce{CO2} absorption for NPV} = 0: \quad \text{approximately 55\%}
\end{equation}

This represents a stringent operating requirement that may be difficult to achieve consistently in practice. Field experience \cite{AFPM2014} suggests packed columns typically operate at 60--80\% \ce{CO2} absorption at the high pH required for reliable \ce{H2S} removal.

\begin{figure}[H]
\centering
\fbox{\begin{tikzpicture}[
    x=0.42cm, y=1cm,
    font=\small,
    every node/.style={inner sep=1pt},
]
  % helpers
  \def\xLabel{-14.5}
  \def\barH{0.40}

  % Base line (22.8% IRR for SCT base case)
  \draw[thick,dashed,gray] (0,0.2) -- (0,7.8);
  \node[above,font=\small] at (0,7.8) {SCT Base: 22.8\%};

  % Hurdle rate line at 15%
  \draw[thick,dotted,AccentC] (-7.8,0.6) -- (-7.8,7.5);
  \node[AccentC,font=\scriptsize,anchor=south] at (-7.8,7.5) {15\% hurdle};

  % ---- SCT Technology (top section) ----
  \node[font=\footnotesize\bfseries,anchor=west] at (\xLabel,7.2) {SCT Static Mixer};

  % Row 1: Diesel benefit
  \fill[AccentA!15] (-5.2,6.2) rectangle (4.9,6.2+\barH);
  \draw[thick,AccentA] (-5.2,6.2) rectangle (4.9,6.2+\barH);
  \node[anchor=east] at (\xLabel,6.2+0.5*\barH) {Diesel benefit (\$/yr)};
  \node[anchor=south, font=\footnotesize] at (-5.2,6.2+\barH) {\$400k};
  \node[anchor=south, font=\footnotesize] at (4.9, 6.2+\barH) {\$600k};

  % Row 2: Capital cost
  \fill[AccentB!15] (-5.4,5.3) rectangle (7.3,5.3+\barH);
  \draw[thick,AccentB] (-5.4,5.3) rectangle (7.3,5.3+\barH);
  \node[anchor=east] at (\xLabel,5.3+0.5*\barH) {Capital cost (TIC)};
  \node[anchor=south, font=\footnotesize] at (-5.4,5.3+\barH) {\$1.2M};
  \node[anchor=south, font=\footnotesize] at (7.3, 5.3+\barH) {\$2.3M};

  % Row 3: CO2 absorption
  \fill[MainBlue!15] (-1.8,4.4) rectangle (1.5,4.4+\barH);
  \draw[thick,MainBlue] (-1.8,4.4) rectangle (1.5,4.4+\barH);
  \node[anchor=east] at (\xLabel,4.4+0.5*\barH) {\ce{CO2} absorption};
  \node[anchor=south, font=\footnotesize] at (-1.8,4.4+\barH) {10\%};
  \node[anchor=south, font=\footnotesize] at (1.5, 4.4+\barH) {2\%};

  % Row 4: NaOH price
  \fill[AccentC!15] (-0.8,3.5) rectangle (0.7,3.5+\barH);
  \draw[thick,AccentC] (-0.8,3.5) rectangle (0.7,3.5+\barH);
  \node[anchor=east] at (\xLabel,3.5+0.5*\barH) {NaOH price (\$/ton)};
  \node[anchor=south, font=\footnotesize] at (-0.8,3.5+\barH) {\$600};
  \node[anchor=south, font=\footnotesize] at (0.7, 3.5+\barH) {\$400};

  % ---- Packed Column (bottom section) ----
  \node[font=\footnotesize\bfseries,anchor=west] at (\xLabel,3.3) {Packed Column (base: 5.3\%)};

  % Row 5: Diesel benefit (packed)
  \fill[AccentA!15] (-4.1,1.8) rectangle (4.0,1.8+\barH);
  \draw[thick,AccentA] (-4.1,1.8) rectangle (4.0,1.8+\barH);
  \node[anchor=east] at (\xLabel,1.8+0.5*\barH) {Diesel benefit};
  \node[anchor=south, font=\footnotesize] at (-4.1,1.8+\barH) {\$400k};
  \node[anchor=south, font=\footnotesize] at (4.0, 1.8+\barH) {\$600k};

  % Row 6: CO2 absorption (packed) - most sensitive
  \fill[MainBlue!15] (-3.5,0.9) rectangle (4.9,0.9+\barH);
  \draw[thick,MainBlue] (-3.5,0.9) rectangle (4.9,0.9+\barH);
  \node[anchor=east] at (\xLabel,0.9+0.5*\barH) {\ce{CO2} absorption};
  \node[anchor=south, font=\footnotesize] at (-3.5,0.9+\barH) {90\%};
  \node[anchor=south, font=\footnotesize] at (4.9, 0.9+\barH) {50\%};

  % X-axis
  \draw[<->,thick] (-13,0.2) -- (13,0.2);
  \foreach \x/\lab in {-13/0, -7.8/15, 0/22.8, 7.8/30, 10.4/35}
    \node[below,font=\footnotesize] at (\x,0.2) {\lab\%};

  \node[below] at (0,-0.25) {Internal Rate of Return (\%)};

  % Annotations
  \node[anchor=west,font=\scriptsize,gray] at (-13,8.1) {Pessimistic};
  \node[anchor=east,font=\scriptsize,gray] at (13,8.1) {Optimistic};

  % Highlight: packed column never exceeds hurdle
  \draw[AccentC, thick, ->] (-11,2.3) -- (-8.5,1.5);
  \node[AccentC, font=\scriptsize, anchor=east, align=right] at (-11,2.5) {Packed column\\never exceeds\\15\% hurdle};

\end{tikzpicture}}
\caption{Tornado diagram comparing IRR sensitivity for SCT (top) and packed column (bottom) technologies. The SCT system maintains IRR above the 15\% hurdle across all scenarios, while the packed column fails to reach the hurdle even in optimistic cases.}
\label{fig:economic_tornado}
\end{figure}


\subsection{Risk-Adjusted Recommendation}

The economic analysis leads to a clear recommendation:

\begin{table}[H]
\centering
\caption{Technology recommendation summary}
\label{tab:recommendation}
\small
\begin{tabularx}{\textwidth}{@{}lYY@{}}
\toprule
\textbf{Factor} & \textbf{Packed Column} & \textbf{SCT System} \\
\midrule
Base case IRR & 5.3\% (below hurdle) & 22.8\% (exceeds hurdle) \\
NPV (15-year, 12\%) & --\$87,000 (value destruction) & +\$1,060,000 (value creation) \\
Downside risk & High (sensitive to \ce{CO2} absorption) & Low (robust economics) \\
Upside potential & Limited & Moderate \\
\midrule
\textbf{Recommendation} & \textbf{Not recommended} & \textbf{Recommended} \\
\bottomrule
\end{tabularx}
\end{table}

\subsection{Validation Requirements}

Before project sanction, the following economic uncertainties require resolution:

\begin{enumerate}
\item \textbf{Vendor quotations:} Obtain budgetary quotes from 2--3 SCT technology providers to validate the \$1.78M TIC estimate.

\item \textbf{Diesel displacement:} Analyse historical fuel consumption records to confirm the \$500,000/yr benefit assumption.

\item \textbf{Selectivity performance:} Review vendor performance guarantees for \ce{CO2} co-absorption limits. Consider pilot testing if guarantees are not available.

\item \textbf{NaOH logistics:} Confirm delivered caustic pricing including freight to the TA-58 site.
\end{enumerate}

The project demonstrates strong economic potential with SCT technology, provided the selectivity performance can be achieved. The packed column option should only be reconsidered if SCT vendors cannot provide acceptable performance guarantees.

