% =========================================================
% Conclusions and Recommendations
% =========================================================

\section{Conclusions and Recommendations}
\label{sec:conclusions}

\subsection{Summary of Findings}

This study demonstrates the technical feasibility of caustic scrubbing for \ce{H2S} removal from the TA-58 sour gas stream. However, the economic analysis reveals a critical insight: \textbf{at the 660,000~SCFD scale, selective \ce{H2S} removal is not optional---it is the determining factor for project viability.}

Two technology options were evaluated:

\begin{itemize}
\item \textbf{Packed column scrubber:} Well-established technology with straightforward operation. Equipment sized at 0.75~m diameter $\times$ 5.5~m height. Caustic consumption of 300--500~ton/yr due to 50--80\% \ce{CO2} co-absorption results in \$150,000--250,000/yr operating cost. At base case assumptions, \textbf{IRR of 5.3\% fails to meet the 15\% corporate hurdle rate.}

\item \textbf{SCT static mixer:} Compact inline design (0.3~m $\times$ 1.5~m) exploiting the $10^6$ kinetic rate difference between \ce{H2S} and \ce{CO2}. Caustic consumption of 46--90~ton/yr (2--10\% \ce{CO2} co-absorption) yields \$23,000--45,000/yr operating cost. At base case assumptions, \textbf{IRR of 22.8\% exceeds the hurdle rate}, delivering NPV of +\$1.06M.
\end{itemize}

\subsection{Technology Recommendation}

Based on the quantitative analysis, \textbf{SCT static mixer technology is recommended} for the TA-58 application. The recommendation is supported by:

\begin{table}[H]
\centering
\caption{Technology selection rationale}
\label{tab:selection_rationale}
\small
\begin{tabularx}{\textwidth}{@{}lYY@{}}
\toprule
\textbf{Factor} & \textbf{Packed Column} & \textbf{SCT Mixer (Recommended)} \\
\midrule
Economic viability & IRR 5.3\% (below hurdle) & IRR 22.8\% (exceeds hurdle) \\
NPV (15-year, 12\%) & --\$87,000 (value destruction) & +\$1,060,000 (value creation) \\
Selectivity mechanism & pH control (operationally challenging) & Contact time (inherent to design) \\
Caustic cost sensitivity & High (±\$100k/yr swing) & Low (±\$15k/yr swing) \\
Capital cost & \$2.53M TIC & \$1.78M TIC (30\% lower) \\
Footprint & 0.75~m $\times$ 5.5~m column & 0.3~m $\times$ 1.5~m inline \\
\bottomrule
\end{tabularx}
\end{table}

The packed column should only be reconsidered if:
\begin{romanlist}
\item SCT vendors cannot provide acceptable performance guarantees for \ce{CO2} co-absorption limits
\item Site personnel have strong preference for conventional equipment despite the economic penalty
\item Pilot testing reveals selectivity performance below vendor claims
\end{romanlist}

\subsection{Key Trade-offs}

The technology selection involves the following considerations:

\begin{table}[H]
\centering
\caption{Trade-off summary}
\label{tab:tradeoffs}
\small
\begin{tabularx}{\textwidth}{@{}lYY@{}}
\toprule
\textbf{Factor} & \textbf{Packed Column Favoured} & \textbf{SCT Mixer Favoured} \\
\midrule
Technology maturity & Decades of field experience & --- \\
Operating cost & --- & 5--10$\times$ lower caustic consumption \\
Capital cost & --- & 30\% lower TIC \\
Footprint & --- & Compact inline installation \\
Control simplicity & More forgiving pH tolerance & --- \\
Economic return & --- & Meets hurdle rate; positive NPV \\
\bottomrule
\end{tabularx}
\end{table}

\textbf{The economic analysis demonstrates that the packed column's advantages in maturity and control simplicity do not compensate for its fundamental economic disadvantage at this scale.}

\subsection{Data Gaps Requiring Resolution}

Before project sanction, the following uncertainties require closure:

\begin{enumerate}
\item \textbf{Vendor quotations:} Budgetary quotes from 2--3 SCT technology providers to validate the \$1.78M TIC estimate. Target vendors include Merichem, Trimeric, and Koch-Glitsch.

\item \textbf{Selectivity guarantee:} Written vendor guarantee for maximum \ce{CO2} co-absorption (target: $<$10\%). If guarantee cannot be obtained, pilot testing should be considered.

\item \textbf{Diesel displacement:} Analysis of historical fuel consumption and engine runtime records to confirm the \$500,000/yr benefit assumption.

\item \textbf{Spent caustic treatment:} MBBR vendor engagement for packaged systems at 43~kg~S/day capacity, or confirmation of chemical oxidation as fallback.

\item \textbf{NaOH logistics:} Delivered caustic pricing including freight to the TA-58 site; storage requirements and delivery frequency.
\end{enumerate}

\subsection{Recommended Next Steps}

\begin{enumerate}
\item \textbf{Vendor engagement (immediate):} Issue RFQ to 2--3 SCT caustic scrubber vendors with process data package including:
  \begin{itemize}
  \item Gas flow rate, composition, and operating conditions
  \item Required \ce{H2S} removal efficiency and outlet specification
  \item Maximum acceptable \ce{CO2} co-absorption (target 5--10\%)
  \item Request for performance guarantee and reference list
  \end{itemize}

\item \textbf{Diesel benefit validation:} Review 12--24 months of operational data to confirm engine runtime and fuel consumption patterns.

\item \textbf{HAZID workshop:} Conduct technology-neutral hazard identification with site operations, maintenance, and HSE stakeholders.

\item \textbf{Pilot testing (if required):} If vendor guarantees are insufficient, plan field pilot at TA-58 using slip-stream from VRU discharge.

\item \textbf{FEED phase:} Following vendor selection, proceed to front-end engineering with selected technology provider.
\end{enumerate}

\subsection{Concluding Statement}

This study provides sufficient technical and economic basis for management decision on project advancement. The analysis demonstrates that:

\begin{romanlist}
\item Selective \ce{H2S} removal using SCT technology is technically feasible and economically attractive
\item Non-selective packed column technology fails to meet economic hurdles at this scale
\item The project creates an estimated \$1.06M NPV over 15 years while enabling diesel displacement
\item The selectivity imperative---achieving high \ce{H2S} removal while minimising \ce{CO2} co-absorption---is the critical success factor
\end{romanlist}

The project is recommended for advancement to vendor engagement and detailed feasibility, with SCT static mixer technology as the preferred configuration.

