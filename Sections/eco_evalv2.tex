\section{Economic Evaluation of Packed Column and SCT Systems}
\label{sec:economic_evaluation}

\subsection{Objective and Estimation Class}

The economic evaluation aims to compare a conventional packed absorption column and a suspended carrier technology (SCT) reactor for \ce{H2S} removal in the presence of \ce{CO2}.
The analysis is conducted at an AACE Class~4 level, corresponding to an early-stage conceptual design with limited engineering definition and an expected accuracy range of approximately $-30\%$ to $+50\%$ \cite{AACE18R97}.

In accordance with recommended practice, equipment costs were estimated using a combination of reference cost data, capacity scaling laws, material factors, and factored installation methods widely applied in chemical engineering design \cite{TowlerSinnott2022b,PetersTimmerhaus2003b,Seider2016}.
The methodology prioritises transparency and traceability: each cost element is linked to a physical design parameter and an identifiable literature source or engineering correlation.

\subsection{Equipment Geometry and Design Basis}

The key geometric parameters of the two systems are:

\begin{itemize}
    \item Packed column: diameter $D = 0.75~\mathrm{m}$, height $H = 5.5~\mathrm{m}$.
    \item SCT reactor: diameter $D = 0.30~\mathrm{m}$, length $L = 1.5~\mathrm{m}$ (inline configuration).
\end{itemize}

The corresponding reactor volumes are:

\begin{equation}
V = \frac{\pi D^2}{4}H,
\end{equation}

yielding:

\begin{align}
V_{\text{packed}} &= 2.43~\mathrm{m^3}, \\
V_{\text{SCT}} &= 0.106~\mathrm{m^3}.
\end{align}

These volumes constitute the primary scaling parameters for capital cost estimation.

\subsection{Capital Cost Estimation Methodology}

The purchased equipment cost (PEC) of each major component was estimated using the generalized scaling relationship:

\begin{equation}
C_i = C_{ref,i}\left(\frac{S_i}{S_{ref,i}}\right)^{n_i} F_{mat} F_{loc},
\label{eq:cost_scaling}
\end{equation}

where $C_i$ is the estimated cost, $C_{ref,i}$ is the reference cost, $S_i$ is the characteristic size parameter, $n_i$ is the scaling exponent, and $F_{mat}$ and $F_{loc}$ are material and location factors, respectively.
Scaling exponents were selected from standard process design references \cite{PetersTimmerhaus2003b,Seider2016}.

A location factor of $F_{loc} = 1.1$--$1.2$ was adopted to account for regional deviations from US Gulf Coast benchmark costs \cite{TowlerSinnott2022b}.
Material factors reflect corrosion resistance requirements imposed by \ce{H2S} and alkaline environments \cite{AFPM2014}.

Total installed cost (TIC) was obtained from PEC using a factored approach:

\begin{equation}
TIC = PEC \cdot (1 + f_{pipe} + f_{elec} + f_{civil} + f_{inst} + f_{I\&C}),
\end{equation}

with factor ranges consistent with AACE and classical plant cost breakdowns \cite{AACE18R97,PetersTimmerhaus2003b}.

\subsection{Packed Column System: Capital Costs}

The packed column shell cost was estimated using reference vertical vessel data reported by Towler and Sinnott \cite{TowlerSinnott2022b}.
A carbon steel reference column of diameter $1.5~\mathrm{m}$ and height $10~\mathrm{m}$ with a typical cost of approximately \$120,000 was adopted as baseline.
Scaling to the actual column volume using Eq.~\eqref{eq:cost_scaling} with exponent $n=0.6$ yielded the carbon steel shell cost, which was subsequently adjusted by a stainless steel material factor of approximately 2.4 to reflect corrosion-resistant construction.

Packing material costs were estimated from industrial price ranges for stainless steel Pall rings reported in packed tower design literature and engineering guidelines \cite{strigle1994,Kolmetz2011}.
The packing volume was derived directly from the column geometry and packing height obtained from mass-transfer design calculations \cite{Onda1968,Flagiello2021}.

Pump and blower costs were estimated using empirical correlations for rotating equipment \cite{PetersTimmerhaus2003b} and typical unit power cost ranges reported in process design literature \cite{Seider2016}.
Additional costs for distributors, supports, and internals were estimated as a fraction of the column shell cost.

\subsection{SCT System: Capital Costs}

The SCT reactor was treated as a low-pressure biological reactor rather than a conventional pressure vessel.
Reactor shell costs were estimated using scaled vessel correlations with an imposed minimum fabrication cost to reflect practical manufacturing constraints.
Carrier media costs were derived from reported price ranges for polymeric biofilm carriers used in MBBR and biotrickling systems \cite{odegaard2006}.

Ancillary equipment, including recirculation pumps, blowers, nutrient dosing systems, and monitoring instrumentation, was costed using typical package system ranges and empirical correlations \cite{Seider2016}.
Due to the small physical scale of the SCT reactor, auxiliary systems constitute a significant fraction of total capital cost.

\subsection{Installed Capital Cost Comparison}

Applying factored installation costs typical of Class~4 estimates, the resulting total installed capital costs were found to differ significantly between the two systems.
The packed column exhibits higher capital intensity due to its larger size and corrosion-resistant construction, whereas the SCT system benefits from compact geometry but requires additional process control infrastructure.

\subsection{Operating Cost Estimation}

Operating costs were divided into chemical consumption, energy usage, and maintenance.

For the packed column, caustic soda consumption was calculated from stoichiometric requirements for \ce{H2S} neutralisation, with allowance for excess alkalinity to ensure stable operation \cite{Chen2001,AFPM2014}.
Unit prices for caustic soda were obtained from contemporary industrial market reports \cite{IMARC_CausticSoda}.

Energy consumption was derived from pump and blower power requirements and multiplied by industrial electricity prices reported in official statistics \cite{EIAEPM_C3}.
Annual maintenance costs were estimated as 2--5\% of installed capital cost, consistent with standard chemical engineering economic practice \cite{PetersTimmerhaus2003b,Seider2016}.

For the SCT system, nutrient consumption rates were estimated from biological sulphide oxidation studies \cite{Borges2025,Trisakti2025}.
Energy requirements were lower than for the packed column due to reduced liquid circulation rates.
Maintenance costs were assumed to be slightly higher in relative terms due to biological process variability.

\subsection{Economic Performance Indicators}

The economic performance of both systems was evaluated using annualised cost and net present value (NPV).
The NPV was calculated according to:

\begin{equation}
NPV = -CapEx + \sum_{t=1}^{N}\frac{(R_t - OpEx_t)}{(1+r)^t},
\end{equation}

where $N$ is the project lifetime and $r$ is the discount rate.
A project lifetime of 15~years and discount rates typical of process engineering investments were adopted \cite{Seider2016}.

The results indicate that the packed column is characterised by higher capital and operating costs driven primarily by chemical consumption and corrosion-resistant materials, whereas the SCT system exhibits substantially lower life-cycle costs but greater uncertainty associated with biological performance.
These findings are consistent with trends reported in recent literature on \ce{H2S} treatment technologies \cite{Pudi2022,Borges2025}.
