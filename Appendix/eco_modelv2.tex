\section{Detailed Cost Calculation and Assumptions}
\label{app:cost_calculation}

\subsection{Reference Equipment Costs}

Reference costs for major equipment were selected from authoritative chemical engineering design sources.
Column shell costs were based on published vessel cost data \cite{TowlerSinnott2022b}, pump costs on empirical correlations \cite{PetersTimmerhaus2003b}, and reactor volumetric costs on environmental engineering literature \cite{odegaard2006}.
All reference values were adjusted using scaling laws and material factors as described in Section~\ref{sec:economic_evaluation}.

\subsection{Scaling Exponents and Size Parameters}

Table~\ref{tab:scaling_exponents} summarises the scaling exponents and characteristic size parameters used in the analysis.

\begin{table}[h]
\centering
\caption{Scaling exponents and characteristic size parameters used in equipment cost estimation.}
\label{tab:scaling_exponents}
\begin{tabular}{lcc}
\hline
Equipment type & Size parameter & Exponent $n$ \\
\hline
Packed column shell & volume & 0.6 \\
Packing material & volume & 1.0 \\
Pumps & power & 0.67 \\
Blowers & power & 0.6 \\
Biological reactor & volume & 0.55--0.65 \\
\hline
\end{tabular}
\end{table}

\subsection{Material and Location Factors}

Material factors were selected based on corrosion and process requirements.
Typical values adopted were 2.0--2.8 for stainless steel and 1.3--1.6 for FRP, consistent with published cost engineering guidance \cite{TowlerSinnott2022b,PetersTimmerhaus2003b}.
A location factor of approximately 1.15 was applied to reflect regional cost differences.

\subsection{Factored Cost Breakdown}

Table~\ref{tab:cost_factors} summarises the installation factors applied to PEC.

\begin{table}[h]
\centering
\caption{Typical installation cost factors for Class~4 process equipment estimates.}
\label{tab:cost_factors}
\begin{tabular}{lc}
\hline
Cost category & Factor (fraction of PEC) \\
\hline
Piping & 0.30--0.50 \\
Electrical & 0.10--0.20 \\
Civil/structural & 0.15--0.30 \\
Instrumentation and control & 0.08--0.15 \\
Installation & 0.40--0.70 \\
\hline
\end{tabular}
\end{table}

\subsection{Operating Cost Derivations}

Caustic soda consumption was derived from stoichiometric and kinetic considerations of \ce{H2S} absorption \cite{Chen2001,Danckwerts1965}.
Energy consumption was calculated from pump and blower power requirements.
Maintenance and replacement costs were annualised using standard chemical engineering economic assumptions \cite{Seider2016}.

\subsection{Uncertainty and Interpretation}

Following AACE guidelines, the resulting capital and operating cost estimates are associated with an expected accuracy range of approximately $-30\%$ to $+50\%$ \cite{AACE18R97}.
The results are therefore suitable for comparative technology assessment and conceptual design decision-making, but not for detailed engineering or procurement.
